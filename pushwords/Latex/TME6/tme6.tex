\documentclass{article}
\usepackage[utf8]{inputenc}
\usepackage{hyperref}
\usepackage{graphicx}

\title{Rendu TME6 : Architecture et Fonctionnalités de l'Application Web "PushWords"}
\author{Haotian XUE \\ Hejun CAO}
\date{\today}

\begin{document}
	
	\maketitle
	
	\section{Description du Thème et des Fonctionnalités Principales}
	\textbf{Thème :} Nous créons un outil d'apprentissage en ligne nommé "PushWords", conçu pour aider les utilisateurs à apprendre et mémoriser du vocabulaire en créant et utilisant des cartes de mots.
	
	\textbf{Fonctionnalités principales :}
	\begin{itemize}
		\item Inscription, connexion et gestion de compte utilisateur.
		\item Création, édition et suppression de cartes de mots.
		\item Tests de connaissances lexicales à travers différents modes d'exercices (jeux de correspondance, questions à trous, etc.).
		\item Visualisation des progrès d'apprentissage et des statistiques.
	\end{itemize}
	
	\section{Choix et Utilisation de l'API Web}
	\textbf{API choisie :} WordsAPI pour l'obtention de définitions, exemples de phrases, et prononciations. URLs officiels : \url{https://www.wordsapi.com/}, \url{https://github.com/NamitS27/WordsAPI?tab=readme-ov-file}
	
	\textbf{Utilisation :}
	\begin{itemize}
		\item Lors de l'ajout d'un nouveau mot à leurs cartes, l'application appelle WordsAPI pour récupérer les détails du mot pour l'utilisateur.
		\item Les données fournies par WordsAPI enrichissent l'expérience d'apprentissage, par exemple, en utilisant des exemples de phrases dans les modes d'exercice.
	\end{itemize}
	
	\section{Description des Fonctionnalités de l'Application}
	L'application offre une interface interactive permettant aux utilisateurs de créer et gérer facilement leurs cartes de mots. Plusieurs modes d'exercice sont disponibles pour renforcer la mémoire par répétition. Les utilisateurs peuvent suivre leurs progrès et réalisations à travers des graphiques et des données statistiques.
	
	\section{Cas d'Utilisation de l'Application}
	\begin{enumerate}
		\item \textbf{Inscription et Connexion :} Les utilisateurs s'inscrivent et se connectent en remplissant les informations nécessaires.
		\item \textbf{Création de Cartes de Mots :} Après connexion, les utilisateurs peuvent créer de nouvelles cartes de mots, qui sont enrichies avec des informations de WordsAPI.
		\item \textbf{Modes d'Exercice :} Les utilisateurs sélectionnent des modes d'exercice pour apprendre, basés sur leur historique d'apprentissage.
		\item \textbf{Suivi des Progrès :} Les utilisateurs consultent leurs statistiques d'apprentissage et rapports de progrès.
	\end{enumerate}
	
	\section{Stockage des Données sur le Serveur}
	Les données stockées incluent :
	\begin{itemize}
		\item Informations d'inscription des utilisateurs.
		\item Détails des cartes de mots.
		\item Enregistrements des activités d'apprentissage et progrès.
	\end{itemize}
	
	\section{Mise à Jour des Données et Appels API}
	Les appels à WordsAPI sont effectués lors de l'ajout ou de la consultation de cartes de mots pour enrichir les informations disponibles. Les activités d'apprentissage sont mises à jour après chaque session d'exercice.
	
	\section{Description du Serveur}
	L'architecture RESTful est adoptée avec les services suivants :
	\begin{itemize}
		\item Service Utilisateur pour l'inscription et la gestion des comptes.
		\item Service de Cartes de Mots pour la gestion des cartes.
		\item Service d'Apprentissage pour le suivi des progrès et la recommandation d'exercices.
	\end{itemize}
	
	\section{Description du Client}
	L'application est conçue comme une application monopage (SPA) avec :
	\begin{itemize}
		\item Pages d'inscription/connexion, d'accueil, de création/édition de cartes, de sélection d'exercices, et de visualisation des progrès.
		\item Interaction avec le serveur via des requêtes AJAX pour une mise à jour dynamique.
	\end{itemize}
	
	\section{Requêtes et Réponses entre le Client et le Serveur}
	Les interactions utilisent les méthodes HTTP (GET, POST, PUT, DELETE) avec des données en format JSON pour les requêtes et les réponses.
	
	\section{Schéma Global du Système}
	Un diagramme illustrant les interactions entre le client, le serveur, la base de données, et l'API externe WordsAPI est présenté.
	
	\begin{figure}[h]
		\centering
		\includegraphics[width=0.8\textwidth]{./images/schéma.png}
		\caption{Les interactions entre le client, le serveur, la base de données, et l'API externe}
		\label{fig:exemple1}
	\end{figure}
	
\end{document}
